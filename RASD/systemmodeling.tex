\chapter{System Modeling} \label{cap:cap4}
\section{UML Diagrams}
\subsection{Use Case Diagrams}
\subsubsection{Event Management}
The use case in figure~\ref{fig:eventusecase} shows how a user can manage his agenda. After he logs into the platform he will see his calendar and being able to see schedule of his events. This event can be modified and so deleted or if the user is the event's owner he can invite other users.
 \begin{center}
 \begin{figure}[H]
    \includegraphics[width=1\textwidth]{../UMLDiagram/use_case/EventManagmentUseCase/EventManagment.png}
    \caption{Event Management Use Case}
     \label{fig:eventusecase}
     \end{figure}
   \end{center}  
The use case in the shows how a user can manage his agenda. After he logs into the platform he will see his calendar and being able to see schedule of his events.This events can be modified and so deleted or if he is the event's owner he can invite other users.This use case show how a user can manage his agenda. After he logs into the platform he will see his calendar and being able to see schedule of his events.This events can be modified and so deleted or if he is the event's owner he can invite other users.
\subsubsection{Invitation}
The use case in figure \ref{fig:invitusecase} explains what the user can do when he receive an invitation to an event from an other user. Once he get the notification he can either see the event's info and accept or decline the invitation.\\\\\\
 \begin{center}
 \begin{figure}[H]
    \includegraphics[width=1\textwidth]{../UMLDiagram/use_case/Invitation/Invitation.png}
    \caption{Invitation Use Case}
     \label{fig:invitusecase}
     \end{figure}
   \end{center}  
\subsubsection{View other users' profiles}
The use case in figure \ref{fig:otherprofileusecase} shows how an user can reach the profile of an other user and view his agenda. After he logs in to the platform he will be able to search an user and see his profile, but he will be authorized to see the other user's calendar only if it is set as public by its owner.   
 \begin{center}
 \begin{figure}[H]
    \includegraphics[width=1\textwidth]{../UMLDiagram/use_case/ViewOtherProfiles/UseCaseDiagram0.png}
    \caption{See Other User Profile Use Case}
     \label{fig:otherprofileusecase}
     \end{figure}
   \end{center}  
\subsubsection{Bad Weather}
The use case in figure \ref{fig:otherprofileusecase} explains how an user can behave when he get notified by the platform in case of bad weather condition. Whenever he receives a bad weather notifications the user can act in two different way. If he's the event's owner he can delete the event or modify the event's location while if he's an event's participant he can only manage his event's participation.
 \begin{center}
 \begin{figure}[H]
    \includegraphics[width=1.1\textwidth]{../UMLDiagram/use_case/BadWeather/BadWeather.png}
    \caption{Bad Weather Use Case}
     \label{fig:badweatherusecase}
     \end{figure}
   \end{center}  
\subsection{Class Diagrams}
Once that the requirements of our platform have been defined we are able to compose the class diagram  of the system,as shown in figure ~\ref{fig:classdiagram}.
\begin{center}
 \begin{figure}
    \includegraphics[width=1\textwidth]{../UMLDiagram/class/WeatherCalClassDiagram/ClassDiagram0.png}
    \caption{Class Diagram}
     \label{fig:classdiagram}
     \end{figure}
   \end{center}  
\subsection{State Diagrams}
\subsubsection{Invitation}
In figure \ref{fig:invstatediagram} the state diagram of an invitation to an event is shown.
 \begin{center}
 \begin{figure}[H]
    \includegraphics[width=1\textwidth ]{../UMLDiagram/InvitationStatechart/InvitationStatechart.png}
    \caption{Invitation State Diagram}
     \label{fig:invstatediagram}
     \end{figure}
   \end{center}
As the event owner invites a user, an invitation is created, having an unknown participation, which is defined by the first state. When an invitation is in this state means that an invitation for a user to an event has been made, but he didn't respond to the invitation or he don't even seen that yet. As the user responds to the notification, the state of the invitation becomes definite, so in can be "Yes", "No" or "Maybe". The user can then change that state among these three states, until the event has ended. Infact, after the end of the event the modification of the participation will be disallowed.
 \subsection{Sequence Diagrams}
\subsubsection{User Registration}
Figure \ref{fig:regseqdiag} shows the process for the registration of a new user into the WeatherCal platform. As a user accesses the system, it will prompt it either to login or to register to the system. If the user chooses the latter this is what happens.
\begin{center}
 \begin{figure}[H]
    \includegraphics[width=1\textwidth]{../UMLDiagram/sequence/UserRegister/UserRegister.png}
    \caption{Sequence diagram of user registration}
     \label{fig:regseqdiag}
     \end{figure}
   \end{center}  
The user will see a registration form in which he will input his data and the credentials he wants to use for accessing the system. As the user submits the form and data is sent to the server the validation occurs. It can happen that, for instance, a new user chooses an already taken nickname, or he registers with an email which is yet on the system. These cases will cause an exception to be thrown and an error to be displayed to the user. However, in this diagram, the case in which the validation succedes is represented. After that, the user data will be stored and the credentials can be used for authentication purposes.
\subsubsection{Login}
\begin{center}
 \begin{figure}[H]
    \includegraphics[width=1\textwidth]{../UMLDiagram/sequence/LoginDiagram/LoginDiagram.png}
    \caption{Login Sequence Diagram}
     \label{fig:logseqdiag}
     \end{figure}
   \end{center}  
\subsubsection{Event creation}
\begin{center}
 \begin{figure}[H]
    \includegraphics[width=1\textwidth]{../UMLDiagram/sequence/CreateEvent/CreateEvent.png}
    \caption{Create Event Sequence Diagram}
     \label{fig:createseqdiag}
     \end{figure}
   \end{center}  
\subsubsection{Event modification}
\begin{center}
 \begin{figure}[H]
    \includegraphics[width=1\textwidth]{../UMLDiagram/sequence/ModifyEvent/ModifyEvent.png}
    \caption{Sequence Diagram of event's modification}
     \label{fig:modseqdiag}
     \end{figure}
   \end{center}  
\subsubsection{Event deletion}
\begin{center}
 \begin{figure}[H]
    \includegraphics[width=1\textwidth]{../UMLDiagram/sequence/DeleteEvent/DeleteEvent.png}
    \caption{Sequence Diagram of event's deletion}
     \label{fig:delseqdiag}
     \end{figure}
   \end{center}  
\subsubsection{Invitation}
\begin{center}
 \begin{figure}[H]
    \includegraphics[width=1\textwidth,height=1\textwidth]{../UMLDiagram/sequence/InvitationDiagram/InvitationDiagram.png}
    \caption{Sequence Diagram of an invitation}
     \label{fig:invitseqdiag}
     \end{figure}
   \end{center}  
\subsubsection{Invite notification}
\begin{center}
 \begin{figure}[H]
    \includegraphics[width=1\textwidth,height=1\textwidth]{../UMLDiagram/sequence/InvitationNotificationDiagram/InvitationNotificationDiagram.png}
    \caption{Sequence Diagram of an invitation's notification}
     \label{fig:notseqdiagr}
     \end{figure}
   \end{center}  
\subsubsection{Modify participation}
\begin{center}
 \begin{figure}[H]
    \includegraphics[width=1\textwidth]{../UMLDiagram/sequence/ModifyParticipation/ModifyParticipation.png}
    \caption{Modify participation Sequence Diagram}
     \label{fig:modpartseqdiag}
     \end{figure}
   \end{center}  
\section{Alloy}
We use Alloy Analyzer to figure out if our Class Diagram was designed in a consistent way so that even our system will be implemented in a consistent style.\\
Below are reported the code used to represent our system and some models generated by the Alloy Analyzer tool.
\subsection{Code}
\singlespacing
\begin{lstlisting}[frame=single,caption=Alloy Analyzer code label=list:alloycode]
//SIGNATURE
sig RegisteredUser{
calendar:one Calendar
}
sig Invitation{
Receiver: some RegisteredUser,
event:one Event
}
sig WeatherConstrain{
}
sig Calendar{
}
sig Event{
 Owner:one RegisteredUser,
 constrain:one WeatherConstrain
}
//FACTS
fact calendarConstrain{
//the number of Calendar must be equals to the RegisteredUser
#Calendar=#RegisteredUser
//two or more RegisteredUser cannot refer to the same calendar
no disj r1,r2:RegisteredUser | r1.calendar=r2.calendar
}
fact eventConstrain{
//the number of Invitation must be equals to Event
#Event=#Invitation
//each event refer to a different weather constrain
#WeatherConstrain=#Event
no disj e1,e2:Event | e1.constrain=e2.constrain
}
fact invitationConstrain{
//two or more Invitation cannot refer to the same event 
no disj r,p:Invitation | p.event=r.event //and p.Receiver = r.Receiver
//the owner of an event cannot receive the invitation of its event
all  disj e:Event , i1: Invitation | i1.event=e implies e.Owner not in i1.Receiver
}
//ASSERTION
assert oneCalendarforUser{
no disj r1,r2:RegisteredUser | r1.calendar=r2.calendar
}
//check oneCalendarforUser
assert  ownerNotReceveir{
all  disj e:Event , i1: Invitation | i1.event=e implies e.Owner not in i1.Receiver 
}
//check ownerNotReceveir
assert oneInvitationforEvent{
no disj r,p:Invitation | p.event=r.event and p.Receiver = r.Receiver
}
//check oneInvitationforEvent
assert onecal{
#Calendar=#RegisteredUser
}
//check cal
assert oneConstrain{
no disj e1,e2:Event | e1.constrain=e2.constrain
}
//check oneConstrain
pred show(){} 
run show 
\end{lstlisting}
\begin{center}
 \begin{figure}[H]
    \includegraphics[width=0.6\textwidth]{../Alloy/checkassert.png}
    \caption{Assertion verified}
     \label{fig:asser}
     \end{figure}
   \end{center} 
\doublespacing
\subsection{Model generated}
In this section are reported some of the various scenarios representing the model of our system.
\subsubsection{Interaction between all users}
\subsubsection{No interaction between the users}
\subsubsection{Interaction only between a few users}








