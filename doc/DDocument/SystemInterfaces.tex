\chapter{System} \label{cap:cap2}

\section{System Description}
The system will implement a calendar as a web application, splitted into client-side and server-side (motivations discussed in \autoref{sec:sec2.2} and \autoref{sec:sec2.3}). The former will be used for implementing the asyncronous facilities delivered in the calendar, while the latter will be used both as an interface for the former to interact with the persistency and for providing web pages to the client.

\section{Design Constraints} \label{sec:sec2.2}
The application, first, will have some constraints on the system proposed by the customer explicitly. The first one is the use of J2EE as server-side application implying the use of a storage for persisting the data (events, users, invitations, etc.). This constraint entails that the client-server architecture will be adopted in the WeatherCal system.\\
Client constraints even include the time for the system development, which is due to January \nth{25}, 2015.\\
Constraints imposed by the client does not include any strict restraint on the hardware and the software over which the system will need to be deployed and any further requirement will be added, giving the possibility to be platform independant. Anyway the system on which the platform will be deployed on will have an impact on the server-side application performances and both the client-side environment and the network connecting the client and the server will impact the client-side application performances. Even though both the client and the server software involved in this project have some requirements on the hardware and the software to be used, so they will make our constraints.

\section{System Architecture} \label{sec:sec2.3}
As said in \autoref{sec:sec2.2} the system will use J2EE for implementing the server-side application and thus the system will rely on a client-server achitecture. The server will also implement the MVC design pattern by means of the Java Server Faces framework, which will facilitate the development of the structure taking advantage of both the design pattern and the facilities brought.\\
Another choice is to implement a web client because, instead of a traditional application, it provides universal access and no need of being in possess of a dedicated client application.\\
The application also needs to persist data, so we decided to use PostgreSQL free RDBMS to accomplish this task.\\
The related client side will be developed using both the Web tier provided by JSF and Marionette.js, a Javascript framework, with its dependecies, for making the client more responsive and interactive.
The graphic environment of the web pages will be managed by PrimeFaces, a JSF component suite. It was chosen among other similar libraries both for its features and performances in combination with the availability of support from the client. We also added OmniFaces for the purpose of gaining advantage of a library of useful functions.