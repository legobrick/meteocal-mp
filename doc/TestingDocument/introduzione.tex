\chapter{Introduction} \label{cap:cap1}
\section{System Overview}
WeatherCal  it's a Web application that allows the registered users to manage their schedule basing on the weather conditions. The users are also able to invite people to their event and to publicize their event. The system is even capable of notifying the users, as they log on to the system, if the weather forecast for the event is not the desired one. The target of this software is, therefore, to give the users a tool for schedule their events smartly, giving the possibility to change preferences as the weather forecast changes.
\section{Purpose}
This document describes the plan for testing the developed system against the user requirements defined in requirements document. So the purpose of this acceptance test is to make sure that the project complies with the requirements of requirement documents.
\section{Scope}
This document includes the plan, items, scope, approach, environment and procedure of WeatherCal acceptance test. After that, the responsibilities of developers and user representatives are identified. At last, risks and contingencies are specified to ensure the test reliability. Other information not related to the test activities is not included in this document.
The document is addressed to either the team members,the  stakeholder and the course staff.
\section{Document Organization}
The document is structured as follows 
\begin{itemize}
\item {\it Introduction}, describes contents of this guide, used documentation during developing process etc.
\item {\it Testing} describes the items to be tested.
\item {\it Testing Case and Specifications}, describes the test cases and their specifications.
\end{itemize}
\section{References}
{\it Requirement Analysis and Specification Document WeatherCal },  Paolo Polidori,Edemanti Marco.\\
{\it Design Document WeatherCal },  Paolo Polidori,Edemanti Marco.

 
 
 
 
 
 